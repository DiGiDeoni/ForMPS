
\documentclass[12pt]{article}
\usepackage{geometry} % see geometry.pdf on how to lay out the page. There's lots.
\usepackage{listings}
\usepackage[usenames,dvipsnames]{color}

\geometry{a4paper} % or letter or a5paper or ... etc
% \geometry{landscape} % rotated page geometry

\lstset{ %
language=Fortran,                % choose the language of the code
basicstyle=\footnotesize,       % the size of the fonts that are used for the code
%numbers=left,                   % where to put the line-numbers
%numberstyle=\footnotesize,      % the size of the fonts that are used for the line-numbers
stepnumber=1,                   % the step between two line-numbers. If it's 1 each line 
                                % will be numbered
numbersep=5pt,                  % how far the line-numbers are from the code
%aboveskip=2pt,
%belowskip=2pt,
backgroundcolor=\color{White},  % choose the background color. You must add \usepackage{color}
showspaces=false,               % show spaces adding particular underscores
showstringspaces=false,         % underline spaces within strings
showtabs=false,                 % show tabs within strings adding particular underscores
frame=single,	                % adds a frame around the code
tabsize=2,	                % sets default tabsize to 2 spaces
captionpos=b,                   % sets the caption-position to bottom
breaklines=true,                % sets automatic line breaking
breakatwhitespace=false,        % sets if automatic breaks should only happen at whitespace
%title=\lstname,                 % show the filename of files included with \lstinputlisting;
                                % also try caption instead of title
escapeinside={\%*}{*)},         % if you want to add a comment within your code
morekeywords={*,...},            % if you want to add more keywords to the set
moredelim=[is][{\itshape \color{Mahogany}}]{|}{|}
}

\title{{\em FortranMPS} and {\em Tensor} reference manual}
\author{Fernando M. Cucchietti}
\date{} % delete this line to display the current date

%%% BEGIN DOCUMENT
\begin{document}

\maketitle


\section{Tensor\_Class}
\subsection{Type Tensor}

\subsubsection{Creators}

\begin{lstlisting}
Tensor# new_Tensor( integer dim1 [, dim2, ...,dim# ] )
		|Creates a random new Tensor of # dimensions, each of length dimX|
Tensor# new_Tensor( integer dim1 [, dim2, ...,dim# ], complex(8) constant )
		|Creates a new Tensor of # dimensions where all entries are constant|
Tensor# new_Tensor( Tensor aTensor )
		|Creates a new Tensor of # dimensions that is a copy of aTensor|
Tensor# new_Tensor( complex(8) data(:,...,:) )
		|Creates a new Tensor of # dimensions from a complex(8) array|
\end{lstlisting}

\subsubsection{Destructor}

\begin{lstlisting}
integer Tensor%Delete()
		|Deletes the tensor and frees all associated memory. Returns an error type|
\end{lstlisting}

\subsubsection{Public methods}

\begin{lstlisting}
logical Tensor%IsInitialized
		|Returns true if tensor is properly initialized|
subroutine Tensor%Print( ?character Message, ?integer error)
		|Prints the contents of the tensor to screen, with optional Message. Optionally an error code can be obtained|
subroutine Tensor%PrintDimensions( ?character Message )
		|Prints the dimensions of the tensor to screen, with optional Message.|
integer(:) Tensor%getDimensions()
		|Returns the dimensions of the tensor in an integer vector|
real(8) Tensor%Norm()
		|Returns the 1-norm of the tensor|
\end{lstlisting}
In this last case the 1-norm is defined as $\sum_{i_1,i_2,...} |t_{i_1,i_2,...}|$.

\subsubsection{Tensor2 extra methods}
\begin{lstlisting}
subroutine Tensor%SVD( Tensor2 U, Tensor2  Sigma, Tensor2  vTransposed, ?integer errorCode )
-- also subroutine SingularValueDecomposition(thisTensor, U, Sigma, vTransposed, ErrorCode)
	|Returns the SVD of the Tensor = U*Sigma*vTransposed.
	Sigma is a diagonal Tensor2.|
Tensor3 Tensor%SplitIndex( Index whichIndex ,integer Partition) 
	|Splits the index (can be FIRST or SECOND) in Partition times. 
	Example: 	(8x6)%SplitIndex(FIRST, 2) => (2x4x6)
			(8x6)%SplitIndex(SECOND, 3) => (8x3x2)|
-- also SplitIndexOfTensor2( Tensor2 thisTensor, Index whichIndex ,integer Partition) 
Tensor2 Tensor%dagger(this) |Transpose-conjugates the tensor|
-- also ConjugateTranspose2
Tensor2 Tensor%CompactFromLeft (Tensor3 aTensor, Index indexToCompact)
-- also Mirror_Compact_Left_With_Tensor3 (Tensor2 thisTensor, Tensor3 aTensor, Index indexToCompact)
Tensor2 Tensor%CompactFromRight (Tensor3 aTensor, Index indexToCompact)
-- also Mirror_Compact_Right_With_Tensor3 (Tensor2 thisTensor, Tensor3 aTensor, Index indexToCompact)
	|These methods implement a typical operation of an MPS chain: contract a matrix with two 3-tensors. See formula and plot below.|
\end{lstlisting}
CompactFromLeft and CompactFromRight above are the following:
\begin{eqnarray}
L'(T)=\sum_{n \in whichIndex} T^\dagger_n L T_n \\
R'(T)=\sum_{n \in whichIndex} T_n R T^\dagger_n 
\end{eqnarray}
where $T_n$ is the Tensor2 extracted from Tensor3 by fixing the index whichIndex to $n$. Schematically, the LeftCompact (L,T) is

\setlength{\unitlength}{0.4cm}
\begin{picture}(11,8)
\thicklines
\put(1,2){\line(0,1){4}}
\put(1,2){\line(1,0){5}}
\put(4,2){\line(0,1){4}}
\put(1,6){\line(1,0){5}}
\put(2,3.5){$L$}
\put(7,2){\circle{2}}
\put(7,6){\circle{2}}
\put(6.5,1.5){$T^*$}
\put(6.5,5.5){$T$}
\put(7,3){\line(0,1){2}}
\put(8,6){\line(1,0){2}}
\put(8,2){\line(1,0){2}}
\end{picture}

and the RightCompact (R,T) is

\setlength{\unitlength}{0.4cm}
\begin{picture}(11,8)
\thicklines
\put(10,2){\line(0,1){4}}
\put(5,2){\line(1,0){5}}
\put(7,2){\line(0,1){4}}
\put(5,6){\line(1,0){5}}
\put(8,3.5){$R$}
\put(4,2){\circle{2}}
\put(4,6){\circle{2}}
\put(3.5,1.5){$T^*$}
\put(3.5,5.5){$T$}
\put(4,3){\line(0,1){2}}
\put(1,6){\line(1,0){2}}
\put(1,2){\line(1,0){2}}
\end{picture}

  \subsubsection{Tensor3 extra methods}
\begin{lstlisting}
Tensor2 JoinIndices (DoubleIndex Index1, DoubleIndex Index2 )
	|Joins two of the indices of Tensor3. Indexn can be FIRST, SECOND, THIRD to leave that index alone, and combinations like THIRDANDFIRST to join those indices|
-- also JoinIndicesOfTensor3
Tensor3 CompactFromBelow (Index Bound3, Tensor4 aTensor, Index Bound4, Index Free4)
	|Sums over one index of Tensor3 and aTensor4, and joins two indices of Tensor4 with the other two indices of Tensor3|
-- also Compact_From_Below_With_Tensor4 (Index Bound3, Tensor4 aTensor, Index Bound4, Index Free4)
\end{lstlisting}
CompactFromBelow is schematically

\setlength{\unitlength}{0.4cm}
\begin{picture}(7,4)
\thicklines
\put(0,3){\line(1,0){2}}
\put(4,3){\line(1,0){2}}
\put(3,0){\line(0,1){2}}
\put(3,3){\circle{2}}
\put(2.5,2.7){${\tilde T}_3$}
\put(3.2,0.5){$F_4$}
\put(0.1,3.2){$\alpha \alpha'$}
\put(4.3,3.2){$\beta \beta'$}
\put(6.5,3){$=$}
\end{picture}
\setlength{\unitlength}{0.4cm}
\begin{picture}(6,9)
\thicklines
\put(0,3){\line(1,0){2}}
\put(4,3){\line(1,0){2}}
\put(0,7){\line(1,0){2}}
\put(4,7){\line(1,0){2}}
\put(3,0){\line(0,1){2}}
\put(3,4){\line(0,1){2}}
\put(3,3){\circle{2}}
\put(3,7){\circle{2}}
\put(2.5,2.7){$T_4$}
\put(2.5,6.7){$T_3$}
\put(3.2,0.5){$F_4$}
\put(0.5,3.2){$\alpha'$}
\put(5,3.2){$\beta'$}
\put(0.5,7.2){$\alpha$}
\put(5,7.2){$\beta$}
\put(3.1,4.2){$B_4$}
\put(1.5,5){$B_3$}
\end{picture}

  \subsubsection{Tensor4 extra methods}
\begin{lstlisting}    
Tensor2 JoinIndices (DoubleIndex Index1, DoubleIndex Index2 )
	|Joins two pairs of the indices of Tensor4. Indexn can be FIRSTANDFOURTH or SECONDANDTHIRD constants. |
-- also JoinIndicesOfTensor4
\end{lstlisting}

\subsection{Operators and methods}
\begin{lstlisting} 
operator(.x.): 
	Usage 		TensorN .x. TensorM
	-- also		TensorN * TensorM
	|Contracts the right-most index of TensorN with the left-most index of TensorM
	Use with T2=T2.x.T2, T1=T2.x.T1, T0=T1.x.T1, T4=T3.x.T3, T3=T2.x.T3|
\end{lstlisting}

\begin{lstlisting} 
operator(.xx.): 
	Usage 		TensorN .xx. TensorM
	-- also		TensorN ** TensorM
	|Contracts the first two indexes of TensorN with the  first two indexes of TensorM
	Use with T0=T2.xx.T2, T4=T4.xx.T4|
\end{lstlisting}

\section{MPSTensor\_Class}

\section{MPS\_Class}

\section{Multiplicator\_Class}
\end{document}




  interface operator (*)
     module procedure &
     	  & number_times_Tensor1,number_times_Tensor2,number_times_Tensor3,number_times_Tensor4, &
     	  & Tensor2_matmul_Tensor2, Tensor2_matmul_Tensor1, Tensor1_matmul_Tensor2, &
     	  & Tensor1_dotProduct_Tensor1, &
     	  & Tensor3_matmul_Tensor3, Tensor2_matmul_Tensor3, Tensor3_matmul_Tensor2
  end interface

  interface operator (**)
     module procedure &
     &   Tensor4_doubletimes_Tensor4,Tensor2_doubletimes_Tensor2
  end interface

  interface operator (.x.)
     module procedure &
          & Tensor2_matmul_Tensor2, Tensor2_matmul_Tensor1, Tensor1_matmul_Tensor2, &
          & Tensor1_dotProduct_Tensor1, Tensor3_matmul_Tensor3, Tensor2_matmul_Tensor3, Tensor3_matmul_Tensor2
  end interface

  interface operator (.xx.)
     module procedure &
     &   Tensor4_doubletimes_Tensor4,Tensor2_doubletimes_Tensor2
  end interface

  interface operator (.xplus.)
     module procedure &
     &   MultAndCollapse_Tensor3_Tensor4
  end interface

  interface operator (+)
     module procedure add_Tensor1,add_Tensor2,add_Tensor3,add_Tensor4,add_Tensor5
  end interface

  interface operator (-)
     module procedure subtract_Tensor1,subtract_Tensor2,subtract_Tensor3,subtract_Tensor4,subtract_Tensor5
  end interface

  interface MultAndCollapse
    module procedure MultAndCollapse_Tensor3_Tensor4
  end interface

  interface assignment (=)
     module procedure new_Tensor1_fromAssignment, new_Tensor2_fromAssignment, &
          & new_Tensor3_fromAssignment, new_Tensor4_fromAssignment, new_Tensor5_fromAssignment
  end interface

  interface operator (.diff.)
     module procedure Difference_btw_Tensors
  end interface

  interface operator (.absdiff.)
     module procedure Difference_btw_Tensors_WithAbsoluteValue
  end interface

  interface operator (.equaldims.)
     module procedure  Tensors_are_of_equal_Shape
  end interface

  interface operator (.equaltype.)
     module procedure Tensors_are_of_equal_Type
  end interface

  interface JoinIndicesOf
  	module procedure JoinIndicesOfTensor3,JoinIndicesOfTensor4
  end interface

  interface SplitIndexOf
    module procedure SplitIndexOfTensor2
  end interface

  interface TensorPad
    module procedure Pad_Tensor2
  end interface

  interface Conjugate
    module procedure ConjugateTensor1,ConjugateTensor2,ConjugateTensor3,ConjugateTensor4, ConjugateTensor5
  end interface

  interface TensorTranspose
    module procedure TensorTranspose2,TensorTranspose3,TensorTranspose4,TensorTranspose5
  end interface

  interface ConjugateTranspose
    module procedure ConjugateTranspose2,ConjugateTranspose3,ConjugateTranspose4,ConjugateTranspose5
  end interface

  interface TensorTrace
    module procedure Tensor2Trace
  end interface

  interface CompactLeft
    module procedure Compact_Tensor3_From_Left_With_Tensor2,Mirror_Compact_Left_With_Tensor3
  end interface

  interface CompactRight
    module procedure Compact_Tensor3_From_Right_With_Tensor2,Mirror_Compact_Right_With_Tensor3
  end interface

  interface CompactBelow
    module procedure Compact_From_Below_With_Tensor4
  end interface
